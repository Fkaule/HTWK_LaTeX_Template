\chapter{Beispiele}

\section{Einbinden von Grafiken}

\begin{figure}[hbtp]
\centering
\includegraphics[width=0.5\textwidth]{example-image-a}
\caption{Einbinden einer Grafik}
\end{figure}

\begin{figure}[hbtp]
\centering
\begin{minipage}{0.45\textwidth}
\includegraphics[width=\textwidth]{example-image-b}
\end{minipage}
\hspace{0.05\textwidth}
\begin{minipage}{0.45\textwidth}
\includegraphics[width=\textwidth]{example-image-c}
\end{minipage}
\caption{Einbinden von zwei Grafiken nebeneinander}
\end{figure}

\section{Verwendung von Formeln}

Einfache Formeln können mit dem \texttt{amsmath}-Paket geschrieben werden:

\begin{equation}
f(x) = x^2
\end{equation}

Spezielle Formeln können mit den Befehlen des \texttt{amsmath}-Pakets geschrieben werden:

\begin{align}
\dot{x} &= f(x,t) \\
\ddot{x} &= \frac{d}{dt} \dot{x}
\end{align}

\section{Erstellen von Tabellen}

Tabellen können mit dem \texttt{tabular}-Umgebung erstellt werden:

\begin{table}[!hbtp]
\begin{tabular}{|l|c|r|}
\hline
Spalte 1 & Spalte 2 & Spalte 3 \\
\hline
Zeile 1 & 100 & 200 \\
Zeile 2 & 300 & 400 \\
\hline
\end{tabular}
\end{table}

komplexere Tabelle:

\begin{table}[!hbtp]
\begin{tabular}{@{} lcccc @{}}
 \toprule
 \hfil\multirow{2}{*}{\textbf{Name}}    & \multicolumn{2}{c}{\textbf{Overhead1}}  & \multicolumn{2}{c}{\textbf{Overhead2}}  \\
                             & Header1  & Header2  & Header3  & Header4  \\
 \midrule
 Name A & \makecell[c]{text oben \\ text unten} & text text & text text & text text \\
 Name B & text text & text text & text text text & text text \\
 \bottomrule
 \end{tabular}
\end{table}

\section{Literaturverweis}

Der komplette Workflow wird nun an zwei verschiedenen Zitierstilen (numeric und alphabetic) dargestellt:

Nur die Quellen aus der Bib-Datei die mit \cite{dirac} referenziert werden, werden auch im Literaturverzeichnis dargestellt. Die Bib-Datei kann dabei also auch viel größer sein und nicht verwendete Quellen werden einfach nicht angezeigt.

Um mehrere Quellen (z.B. label1, label2 und label3) zu zitieren werden diese per Komma in \cite{dirac,einstein,knuth-fa} Befehl getrennt:

Um auf eine bestimmte Seite innerhalb der Quelle zu referenzieren kann dies mit eckigen Klammern als Option an den \cite[S. 8]{einstein} Befehl verwendet werden:

\section{Symbolverzeichnis}

ToDo...

