%=======================
%= DOKUMENTENDEFINIION =
%=======================

%---------------
% DOKUMENTENART
%---------------

\documentclass[
a4paper,		% Papiergröße
12pt,			% Schriftgröße
onecolumn,		% Einspaltiger Satz
oneside,		% Einseitiger Satz
openright,		% Kapitelbeginn nur auf der rechten Seite
parskip,		% Legt den Abstand zwischen den nachfolgenden Absätzen fest
liststotoc,		% Tabellen und Abbildungsverzeichnis ohne Nummerierung
bibtotocnumbered,	% Literaturverzeichnis nummerieren und aufführen
appendixprefix,		% Anhang vor Anhang-Nummerierung
chapterprefix=true,	% Kapitel + Nr. vor dem Kapitelnamen  
]{scrreprt}		% Standartklasse für Erstellung von Studienarbeiten


%--------------
% ZUSATZPAKETE
%--------------

\usepackage[left=3cm, right=2cm, top=2cm, bottom=2cm]{geometry} % Seitenlayout
\usepackage{amsmath}            % Formeln
\usepackage[ngerman]{babel}     % Neue deutsche Silbentrennung
\usepackage[T1]{fontenc}        % Korrektes Trennen von Wörtern mit Umlauten und Akzenten.
\usepackage{graphicx}	        % Abbildungen
\usepackage[utf8]{inputenc}     % Zeichenkodierung für UTF-8 (Unicode) falls Editor es unterstützt (bevorzugt)
% \usepackage[latin1]{inputenc} % Zeichenkodierung unter Windows für Umlaute und Sonderzeichen falls kein UTF-8 im Editor
\usepackage{caption}			% Beschriftungen (um Anhangsdaten im Abb/Tab-verz. zu löschen)
\usepackage{booktabs}			% für besseren Tabellen mit top,mid und bottomrule
\usepackage{multirow}			% Spaltenzusammenfassung in der Tabelle
\usepackage{setspace}           % für singlespacing Umgebung
\usepackage{makecell}           % für Tabellen
\usepackage{tabularx}           % für Tabellen
\usepackage{csquotes}           % weil Overleaf es gesagt hat
\usepackage[
backend=biber,
style=numeric,
sorting=none
]{biblatex}                     % Literatur

%--------------------
% Literatur
%--------------------

% Literaturdatenbank (Datei mit Einträgen)
\addbibresource{bibliography.bib}

% Einstellungen um URLs auszublenden
\ExecuteBibliographyOptions{
isbn=false, %keine isbn anzeigen
url=false %keine url anzeigen
}

% MISC in ONLINE ändern da dies oft von Zotero falsch einsortiert wird
\DeclareSourcemap{
  \maps[datatype=bibtex, overwrite=true]{
    \map{
      \step[typesource=misc, typetarget=online]
    }
  }
}

%--------------------
% FORMATDEFINITIONEN
%--------------------

\linespread{1.5}\selectfont			% Allgemeiner Zeilenabstand
\numberwithin{equation}{chapter}		% Zweistzufige Formelnummerierung
\clubpenalty = 10000				% Unterdrücken von Schusterjungen und Hurenkindern
\widowpenalty = 10000
\displaywidowpenalty = 10000
\makeindex					% Programm zur Indexerstellung (Sortierung und Formatierung von Eintraegen)
\addtokomafont{disposition}{\rmfamily} 		% Serifen-Schrift für Überschriften
\setcounter{secnumdepth}{2}			% Tiefe der Nummerierung im Inhaltsverzeichnis

%========================
%= BEGINN DES DOKUMENTES =
%========================

\begin{document}

%----------------------------
% DECKBLATT, AUFGABENSTELLUNG & VERZEICHNISSE ...
%----------------------------

\pagestyle{empty}
\thispagestyle{empty}
\begin{singlespacing}				% Beginnen einfacher Zeilenabstand
\begin{center}
\begin{verbatim}
\end{verbatim}

\raisebox{0pt}{\includegraphics[height=0.5cm]{Abbildungen/0_Logo_Unternehmen.png}}\hspace{\fill}
\raisebox{0pt}{\includegraphics[height=0.7cm]{Abbildungen/0_Logo_HTWK.png}}

\vspace{15mm}
\Large{Hochschule für Technik, Wirtschaft und Kultur Leipzig}\\
\large{Fakultät Ingenieurswissenschaften}\\
\vspace{25mm}


\LARGE{\textbf{Titel der Arbeit}}\\\vspace{15mm}
\Large{Masterarbeit}\\
{\normalsize
von}\\
\vspace{25mm}
{\normalsize 
\begin{tabular}{ll}
Name	&	Mustermann	\\
Vorname &	Max	\\
Matrikelnummer	&	12345	\\
Verantwortlicher Hochschulprofessor	&	Prof. Dr.-Ing. Max Mustermann	\\
Betrieblicher Betreuer	&	Prof. Dr.-Ing. Max Mustermann	\\
\end{tabular}
}\\
\vspace{20mm}
{\normalsize Leipzig, den \today}
\end{center}
\end{singlespacing}				% Beginnen einfacher Zeilenabstand
			% Deckblatt
\include{00_Aufgabenstellung}
\begin{singlespacing}
\clearpage
\begingroup
  \renewcommand*{\chapterpagestyle}{empty}
  \pagestyle{empty}
  \tableofcontents
  \clearpage
\endgroup
	% Inhaltsverzeichnis
\end{singlespacing}
\setcounter{page}{1} 				% Hier mit Seitenzahlen beginnen

% Symbolverzeichnis
\addcontentsline{toc}{chapter}{Symbolverzeichnis}
\clearpage
\begingroup
\renewcommand*{\chapterpagestyle}{empty}
\pagestyle{empty}

ToDo...

\clearpage
\endgroup
	% Symbolverzeichnises

%----------------------------
% BEISPIELE FÜR VERSCHIEDENE FUNKTIONEN (AM ENDE WIEDER LÖSCHEN!)
%----------------------------

% Beispiele
\chapter{Beispiele}

\section{Einbinden von Grafiken}

\begin{figure}[hbtp]
\centering
\includegraphics[width=0.5\textwidth]{example-image-a}
\caption{Einbinden einer Grafik}
\end{figure}

\begin{figure}[hbtp]
\centering
\begin{minipage}{0.45\textwidth}
\includegraphics[width=\textwidth]{example-image-b}
\end{minipage}
\hspace{0.05\textwidth}
\begin{minipage}{0.45\textwidth}
\includegraphics[width=\textwidth]{example-image-c}
\end{minipage}
\caption{Einbinden von zwei Grafiken nebeneinander}
\end{figure}

\section{Verwendung von Formeln}

Einfache Formeln können mit dem \texttt{amsmath}-Paket geschrieben werden:

\begin{equation}
f(x) = x^2
\end{equation}

Spezielle Formeln können mit den Befehlen des \texttt{amsmath}-Pakets geschrieben werden:

\begin{align}
\dot{x} &= f(x,t) \\
\ddot{x} &= \frac{d}{dt} \dot{x}
\end{align}

\section{Erstellen von Tabellen}

Tabellen können mit dem \texttt{tabular}-Umgebung erstellt werden:

\begin{table}[!hbtp]
\begin{tabular}{|l|c|r|}
\hline
Spalte 1 & Spalte 2 & Spalte 3 \\
\hline
Zeile 1 & 100 & 200 \\
Zeile 2 & 300 & 400 \\
\hline
\end{tabular}
\end{table}

komplexere Tabelle:

\begin{table}[!hbtp]
\begin{tabular}{@{} lcccc @{}}
 \toprule
 \hfil\multirow{2}{*}{\textbf{Name}}    & \multicolumn{2}{c}{\textbf{Overhead1}}  & \multicolumn{2}{c}{\textbf{Overhead2}}  \\
                             & Header1  & Header2  & Header3  & Header4  \\
 \midrule
 Name A & \makecell[c]{text oben \\ text unten} & text text & text text & text text \\
 Name B & text text & text text & text text text & text text \\
 \bottomrule
 \end{tabular}
\end{table}

\section{Literaturverweis}

Der komplette Workflow wird nun an zwei verschiedenen Zitierstilen (numeric und alphabetic) dargestellt:

Nur die Quellen aus der Bib-Datei die mit \cite{dirac} referenziert werden, werden auch im Literaturverzeichnis dargestellt. Die Bib-Datei kann dabei also auch viel größer sein und nicht verwendete Quellen werden einfach nicht angezeigt.

Um mehrere Quellen (z.B. label1, label2 und label3) zu zitieren werden diese per Komma in \cite{dirac,einstein,knuth-fa} Befehl getrennt:

Um auf eine bestimmte Seite innerhalb der Quelle zu referenzieren kann dies mit eckigen Klammern als Option an den \cite[S. 8]{einstein} Befehl verwendet werden:

\section{Symbolverzeichnis}

ToDo...



%----------------------------
% EIGENTLICHE INHALT
%----------------------------

\chapter{Einleitung}

%% Die Einleitung dient der Einführung in das Thema und das Themengebiet. Es ist häufig viel zu kurz in studentischen Arbeiten verfasst. Bilder/Diagramme zur Erläuterung sollen verwendet werden, ja sogar eine Zusammenfassung zum Stand der Technik, der dann die Aufgabenstellung motiviert, die selbst noch einmal kurz erläutert wird häufig wird der Inhalt der Arbeit noch einmal kurz zusammengefasst - Was kommt in den nächsten Kapiteln? Umfang von 2-4 Seiten durchaus erwünscht
\chapter{Grundlagen}

%Hier können noch mal die notwendigen grundlegenden Zusammenhänge des Themengebietes dargestellt werden. Formal gesehen ist das überflüssig, da es ja Stand der Technik ist und somit den Leser*innen zugänglich. Aber wenn man ehrlich ist, ist man nicht überall Experte und es ist hilfreich, wenn einem noch mal einige Dinge erläutert werden, die nicht in der allgemeinen Ingenieurausbildung in der Tiefe vorhanden sind.
% beschränken Sie es auf das Notwendigste - einfaches Lehrbuchwissen ist meist fehl am Platz, Zusammenhänge aus der Spezialliteratur nicht
% sollte nicht mehr als 20% des Umfangs der Arbeit ausmache
% Referenzieren Sie sauber die Literatur!
\chapter{Material und Methoden}

%% Den Lesern muss erklärt werden, wie man bestimmte Versuche/Experimente/Modelle aufgebaut und durchgeführt hat, damit man verstehen kann, was im Details gemacht wurde und worauf man achtet, wenn man es ebenso durchführen möchte
%% nochmal: es muss so erklärt werden, dass es ein anderer (Wissenschaftler/Ingenieur) ebenso aufbauen und durchführen kann! Wissenschaftliche Erkenntnis wird erst dann bestätigt, wenn sie wiederholbar ist. Daher in diesem Kapitel exakt arbeiten, ohne ins Geschwafel zu verfallen.
%% hier können auch Literaturreferenzen zu Methodiken und Geräten/Software verwendet werden
\chapter{Ergebnisse}

%% Hier beginnt das Herzstück Ihrer Arbeit. Hier werden Ihre, allein Ihre Ergebnisse vorgestellt.
%% Hier wird nicht bewertet! … Ja, ich weiß, jetzt denkt man, was schreibt man denn da? “Nicht Bewerten” heißt nicht “Nicht Beschreiben”. Sie beschreiben die Ergebnisse, so dass der Leser durch diese von Ihnen geführt wird. Ein Diagramm beinhaltet sich alle wichtigen Erkenntnisse, aber nicht jeder wird diese sofort genauso erkennen wie Sie. Daher beschreiben Sie, was Sie in den Ergebnissen/Diagrammen sehen. Dabei können Sie auch Ausreißer oder Gründe für bestimmte Verläufe erklären. Aber Sie bewerten nicht, z.B. sie sagen hier noch nicht, ob die Ergebnisse auch zu den Erkenntnissen anderer Autoren passen (dies kommt in der Diskussion, nächstes Kapitel)
%% in diesem Kapitel keine Literaturreferenzen mehr - hier gibt es nur Ihre Ergebnisse!
%% man findet häufig in der Literatur, dass Autoren da Kapitel “Ergebnisse und Diskussion” nennen und auch so die Inhalte füllen. Dieses Vermischen klingt erst einmal logischer und ist auch einfacher zu schreiben (für Anfänger), aber es macht die Trennung zwischen Ihren Ergebnissen und der Bewertung dieser schwieriger, da dies unterschiedliche Ziele verfolgt.
\chapter{Diskussion}

%% gehört ebenso zum Herzstück Ihrer Arbeit
%% Jetzt kommt die Bewertung Ihre Ergebnisse:
%% Vergleichen Sie diese zur Literatur (schön mit Referenzen arbeiten, wahrscheinlich einige aus der Einleitung den Grundlagen),
%% Wie passt das in die Theorie/Modellvorstellung zum Problem?
%% Welche Konsequenzen ergeben sich aus den Ergebnissen? Was kann man daraus schlussfolgern? Neue Modellansätze, etc.
%% Gliedern Sie diese Kapitel in weitere Überschriften, damit man die Themen klar erkennen kann. Dabei müssen Sie nicht genau der Beschreibung der Ergebnisse folgen, sondern können auch “quer” zu den Ergebnissen gliedern, so wie es am besten gelesen und verstanden werden kann.
%% Hier dürfen sie (auch) “spinnen”. Teilen Sie Ihre Gedanken/Theorien dem Leser mit, damit dieser damit weiterarbeiten kann. Es muss wissenschaftlich sauber und nachvollziehbar sein, aber es darf auch mit Mut geschlussfolgert und in die Glaskugel interpretiert sein.
\chapter{Zusammenfassung und Ausblick}

%% Die Überschrift sagt eigentlich schon alles: kurz das Problem, Vorgehen und Ergebnisse zusammenfassen und einen Ausblick geben
%% Der Ausblick ist ganz wichtig. Zeigen Sie auf, welche nächsten Schritte getan werden sollten/müssen. Wiederum, lassen die Leser an Ihren Gedanken zum Thema teilhaben
%% hier kommen keine Referenzen, Abbildungen oder ähnliches, nur Fließtext in Absätzen strukturiert
\printbibliography[heading=bibintoc]

%----------------------------
% ANHANG
%----------------------------
\appendix
\chapter*{Anhang}
\thispagestyle{empty}				% keine Anzeige Seitenzahlen und Kapitelüberschriften
\addcontentsline{toc}{chapter}{Anhang}
\section*{Nicht verwendete Werkstoffwerte}

\begin{table}[h!]
\centering
\begin{tabular}{|c|c|c|}
\hline
Werkstoff & Dichte ($kg/m^3$) & Festigkeit (MPa) \\
\hline

Aluminium & 2700 & 300 \\
Stahl & 7850 & 500 \\
Titan & 4500 & 900 \\
Kupfer & 8960 & 340 \\
\hline
\end{tabular}
\caption{Werkstoffkennwerte}
\label{tab:werkstoffkennwerte}
\end{table}			% Anhang immer nur mit sections!

\end{document}
