\chapter{Ergebnisse}

%% Hier beginnt das Herzstück Ihrer Arbeit. Hier werden Ihre, allein Ihre Ergebnisse vorgestellt.
%% Hier wird nicht bewertet! … Ja, ich weiß, jetzt denkt man, was schreibt man denn da? “Nicht Bewerten” heißt nicht “Nicht Beschreiben”. Sie beschreiben die Ergebnisse, so dass der Leser durch diese von Ihnen geführt wird. Ein Diagramm beinhaltet sich alle wichtigen Erkenntnisse, aber nicht jeder wird diese sofort genauso erkennen wie Sie. Daher beschreiben Sie, was Sie in den Ergebnissen/Diagrammen sehen. Dabei können Sie auch Ausreißer oder Gründe für bestimmte Verläufe erklären. Aber Sie bewerten nicht, z.B. sie sagen hier noch nicht, ob die Ergebnisse auch zu den Erkenntnissen anderer Autoren passen (dies kommt in der Diskussion, nächstes Kapitel)
%% in diesem Kapitel keine Literaturreferenzen mehr - hier gibt es nur Ihre Ergebnisse!
%% man findet häufig in der Literatur, dass Autoren da Kapitel “Ergebnisse und Diskussion” nennen und auch so die Inhalte füllen. Dieses Vermischen klingt erst einmal logischer und ist auch einfacher zu schreiben (für Anfänger), aber es macht die Trennung zwischen Ihren Ergebnissen und der Bewertung dieser schwieriger, da dies unterschiedliche Ziele verfolgt.